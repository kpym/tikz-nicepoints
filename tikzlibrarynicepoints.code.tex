% Copyright 2015 by Kpym
%
% This file may be distributed and/or modified
%
% 1. under the LaTeX Project Public License and/or
% 2. under the GNU Public License.
%
% How to use it (by examples) :
%
%   [point] draws a point whose size is dependant of the current line width, whose "border" color is the salme as the text color,
%
%   - custom size :
%     [thick,point] or [point=thick] set the point to the size corresponding to "thick" line
%     [point size=4pt,point] or [point={point size=4pt}] set the point to the size corresponding to 4pt line
%
%   - custom color :
%     [red,point] or [point=red] draw the point in red
%     [point fill=yellow, point] or [point={point fill=yellow}] fill the point in yellow (the default is white)
%
%   [point="A"] makes 3 things :
%       1) same as [point]
%       2) name the point coordinates as (A)
%       3) put the label "$A$" next to the point
%
%   [point="A"*] or [point="A"'] makes the same 1) and 2), but without 3). This is equivalent to [point]coordinate(A).
%
%   [point="A"blue] or [point="A"{blue,below}] set the style of the lable "$A$" to [blue] or [blue,below]
%
%   [point={label=$B$}] is equivalent making 1) and 3), without 2).
%
%   node[point,above]{$B$} is the same as [point] node[above]{$B$}
%
%   \point["A"red] at (1,1); is the same as \path (1,1) [point="A"red];
%
%   If you use a special font whose makes the points ugly, you can reset the point font by
%     [lmpoint] (here "lm" is for "Latin Modern").
\typeout{tikz-nicepoints v1.1}

% some additional libraries used by this one
\usetikzlibrary{quotes}

% This library puts the points on the "points" layer
\pgfdeclarelayer{points}
\pgfsetlayers{main,points}

% All styles
\tikzset{
  /nice points/set point size/.code={\pgfmathsetmacro{\pointsize}{sqrt(\pgflinewidth)}},
  point size/.style={/nice points/set point size/.prefix style={line width=#1}},
  /nice points/reset dot/.style = {inner sep=0, outer sep=0,font=},
  /nice points/outer dot/.style = {/nice points/reset dot, scale=2.5*\pointsize,every dot/.try},
  /nice points/inner dot/.style = {/nice points/reset dot, scale=\pointsize,every dot/.try, text=#1}, /nice points/inner dot/.default={white},
  point fill/.style = {/nice points/inner dot/.default={#1}},
  point coordinate/.style={insert path={coordinate(#1)}},
  quotes mean point/.style={
    node quotes mean/.try={point coordinate=##1,
      label={[direction shorthands, every label quotes/.try, ##2,
        node contents=\ensuremath{##1}, empty label/.try]}
    },
    '/.style={empty label/.style={node contents=}},
    */.style='
  },
  point/.style={quotes mean point,
    insert path={
      node[inner sep=0, overlay, every point/.try, #1, /nice points/set point size,
        path picture={
          \begin{pgfonlayer}{points}
            \node[/nice points/outer dot]{.} node[/nice points/inner dot]{.};
          \end{pgfonlayer}
        }]{}
    }
  },
  lmpoint/.style={
    /nice points/every dot/.style={inner sep=0, outer sep=0,
      node font=\usefont{T1}{lmr}{m}{n}\fontsize{10pt}{0pt}\selectfont}
  }
}

% The following macro allows you tu use \point["A"] at (1,1);
\def\point[#1]#2at#3(#4){\path (#4) [point={#1}]}
